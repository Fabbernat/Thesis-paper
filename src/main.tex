%! Author = Bernát
%! Date = 2025. 10. 29.
% LaTeX mintafájl szakdolgozat és diplomamunkáknak az
% SZTE Informatikai Tanszekcsoportja által megkövetelt
% formai követelményeinek megvalósításához
% Modositva: 2011.04.28 Nemeth L. Zoltan
% A fájl használatához szükséges a magyar.ldf 2005/05/12 v1.5-ös vagy késõbbi verziója
% ez letölthetõ a http://www.math.bme.hu/latex/ weblapról, a magyar nyelvû szedéshez
% Hasznos információk, linekek, LaTeX leirasok a www.latex.lap.hu weboldalon vannak.
%


\documentclass[12pt]{report}

%Magyar nyelvi támogatás (Babel 3.7 vagy késõbbi kell!)
\usepackage[utf8]{inputenc}
\usepackage{t1enc}
\usepackage[magyar]{babel}
% A formai kovetelmenyekben megkövetelt Times betûtípus hasznalata:
\usepackage{times}

%Az AMS csomagjai
\usepackage{amsmath}
\usepackage{amssymb}
\usepackage{amsthm}

%A fejléc láblécek kialakításához:
\usepackage{fancyhdr}

%Természetesen további csomagok is használhatók,
%például ábrák beillesztéséhez a graphix és a psfrag,
%ha nincs rájuk szükség természetesen kihagyhatók.
\usepackage{graphicx}
\usepackage{psfrag}

%tetelszerû környezetek definiálhatók, ezek most fejezetenkent egyutt szamozodnak, pl.
\newtheorem{tet}{tetel}[chapter]
\newtheorem{defi}[tet]{Definíció}
\newtheorem{lemma}[tet]{Lemma}
\newtheorem{áll}[tet]{Állítás}
\newtheorem{köv}[tet]{Következmény}

%Ha a megjegyzések és a példak szövegét nem akarjuk dõlten szedni, akkor
%az alábbi parancs után kell õket definiální:
\theoremstyle{definition}
\newtheorem{megj}[tet]{Megjegyzés}
\newtheorem{pld}[tet]{Példa}

%Margók:
\hoffset -1in
\voffset -1in
\oddsidemargin 35mm
\textwidth 150mm
\topmargin 15mm
\headheight 10mm
\headsep 5mm
\textheight 237mm


% Document
\begin{document}

%A FEJEZETEK KEZDÕOLDALAINAK FEJ ES LÁBLÉCE:
%a plain oldalstílust kell átdefiniálni, hogy ott ne legyen fejléc:
\fancypagestyle{plain}{%
%ez mindent töröl:
\fancyhf{}
% a láblécbe jobboldalra kerüljön az oldalszám:
\fancyfoot[R]{\thepage}
%elválasztó vonal sem kell:
\renewcommand{\headrulewidth}{0pt}
}

%A TÖBBI OLDAL FEJ ÉS LÁBLÉCE:
\pagestyle{fancy}
\fancyhf{}
\fancyhead[L]{A diplomamunka címe}
\fancyfoot[R]{\thepage}


%A címoldalra se fej- se lábléc nem kell:
\thispagestyle{empty}

\begin{center}
\vspace*{1cm}
{\Large\bf Szegedi Tudományegyetem}

\vspace{0.5cm}

{\Large\bf Informatikai Intézet}

\vspace*{3.8cm}


{\LARGE\bf A diplomamunka címe}


\vspace*{3.6cm}

{\Large Diplomamunka}
% vagy {\Large Szakdolgozat}

\vspace*{4cm}

%Értelemszerûen megváltoztatandó:
{\large
\begin{tabular}{c@{\hspace{4cm}}c}
\emph{Készítette:}     &\emph{Témavezetõ:}\\
\bf{Hallgató Claudia}  &\bf{Oktató Bonifác}\\
informatika szakos     &egyetemi docens\\
hallgató&
\end{tabular}
}

\vspace*{2.3cm}

{\Large
Szeged
\\
\vspace{2mm}
2011
}
\end{center}


%A tartalomjegyzék:
\tableofcontents

%A \chapter* parancs nem ad a fejezetnek sorszámot
\chapter*{Feladatkiírás}
%A tartalomjegyzékben mégis szerepeltetni kell, mint szakasz(section) szerepeljen:
\addcontentsline{toc}{section}{Feladatkiírás}

A témavezetõ által megfogalmazott feladatkiírás. Önálló oldalon szerepel.

\chapter*{Tartalmi összefoglaló}
\addcontentsline{toc}{section}{Tartalmi összefoglaló}

A tartalmi összefoglalónak tartalmaznia kell (rövid, legfeljebb egy oldalas, összefüggõ megfogalmazásban)
a következõket: a téma megnevezése, a megadott feladat megfogalmazása - a feladatkiíráshoz viszonyítva-,
a megoldási mód, az alkalmazott eszközök, módszerek, az elért eredmények, kulcsszavak (4-6 darab).

Az összefoglaló nyelvének meg kell egyeznie a dolgozat nyelvével. Ha a dolgozat idegen nyelven készül,
magyar nyelvû tartalmi összefoglaló készítése is kötelezõ (külön lapon), melynek terjedelmét a TVSZ szabályozza.


\chapter*{Bevezetés}
\addcontentsline{toc}{section}{Bevezetés}

Itt kezdõdik a bevezetés, mely nem kap sorszámot.



\chapter{Egy találó cím}

Ez pedig már az elsõ fejezet, ...

\section{Alcím}
Ebben alfejezetek is lehetnek

\subsection{Al-al cím}
Sõt al-al fejezetek is.

\subsection{Másik}
Na lássunk egy másodikat is.

\subsection{Harmadik}
Meg egy harmadikat is.

\section{Mindjárt vége a fejezetnek}
Tényleg, itt valóban vége.


\chapter{Hosszú}
\section{Részletek}
Ebbe a fejezetbe pedig írunk sok sok szöveget. Szöveg, szöveg, szöveg,  szöveg, szöveg, szöveg,   szöveg, szöveg, szöveg
szöveg, szöveg, szöveg,  szöveg, szöveg, szöveg, szöveg, szöveg, szöveg, szöveg, szöveg, szöveg, szöveg, szöveg, szöveg,
szöveg, szöveg, szöveg, szöveg, szöveg, szöveg, szöveg, szöveg, szöveg, szöveg, szöveg, szöveg, szöveg, szöveg, szöveg,
szöveg, szöveg, szöveg, szöveg, szöveg, szöveg, szöveg, szöveg, szöveg, szöveg, szöveg, szöveg,
szöveg, szöveg, szöveg, szöveg, szöveg, szöveg, szöveg, szöveg, szöveg, szöveg, szöveg, szöveg, szöveg, szöveg, szöveg,
szöveg, szöveg, szöveg, szöveg, szöveg, szöveg, szöveg, szöveg, szöveg, szöveg, szöveg, szöveg, szöveg, szöveg, szöveg,
szöveg, szöveg, szöveg, szöveg, szöveg, szöveg, szöveg, szöveg, szöveg, szöveg, szöveg, szöveg, szöveg, szöveg, szöveg,
szöveg, szöveg, szöveg, szöveg, szöveg, szöveg, szöveg, szöveg, szöveg, szöveg, szöveg, szöveg, szöveg, szöveg, szöveg,
szöveg, szöveg, szöveg, szöveg, szöveg, szöveg, szöveg, szöveg, szöveg, szöveg, szöveg, szöveg, szöveg, szöveg, szöveg,
szöveg, szöveg, szöveg, szöveg, szöveg, szöveg, szöveg, szöveg, szöveg, szöveg, szöveg, szöveg, szöveg, szöveg, szöveg,
szöveg, szöveg, szöveg, szöveg, szöveg, szöveg, szöveg, szöveg, szöveg, szöveg, szöveg, szöveg, szöveg, szöveg, szöveg,
szöveg, szöveg, szöveg,  szöveg, szöveg, szöveg, szöveg, szöveg, szöveg, szöveg, szöveg, szöveg, szöveg, szöveg, szöveg,
szöveg, szöveg, szöveg, szöveg, szöveg, szöveg, szöveg, szöveg, szöveg, szöveg, szöveg, szöveg, szöveg, szöveg, szöveg,
szöveg, szöveg, szöveg, szöveg, szöveg, szöveg, szöveg, szöveg, szöveg, szöveg, szöveg, szöveg,
szöveg, szöveg, szöveg, szöveg, szöveg, szöveg, szöveg, szöveg, szöveg, szöveg, szöveg, szöveg, szöveg, szöveg, szöveg,
szöveg, szöveg, szöveg, szöveg, szöveg, szöveg, szöveg, szöveg, szöveg, szöveg, szöveg, szöveg, szöveg, szöveg, szöveg,
szöveg, szöveg, szöveg, szöveg, szöveg, szöveg, szöveg, szöveg, szöveg, szöveg, szöveg, szöveg, szöveg, szöveg, szöveg,
szöveg, szöveg, szöveg, szöveg, szöveg, szöveg, szöveg, szöveg, szöveg, szöveg, szöveg, szöveg, szöveg, szöveg, szöveg,
szöveg, szöveg, szöveg, szöveg, szöveg, szöveg, szöveg, szöveg, szöveg, szöveg, szöveg, szöveg, szöveg, szöveg, szöveg,
szöveg, szöveg, szöveg, szöveg, szöveg, szöveg, szöveg, szöveg, szöveg, szöveg, szöveg, szöveg, szöveg, szöveg, szöveg,
szöveg, szöveg, szöveg, szöveg, szöveg, szöveg, szöveg, szöveg, szöveg, szöveg, szöveg, szöveg, szöveg, szöveg, szöveg,
szöveg, szöveg, szöveg,  szöveg, szöveg, szöveg, szöveg, szöveg, szöveg, szöveg, szöveg, szöveg, szöveg, szöveg, szöveg,
szöveg, szöveg, szöveg, szöveg, szöveg, szöveg, szöveg, szöveg, szöveg, szöveg, szöveg, szöveg, szöveg, szöveg, szöveg,
szöveg, szöveg, szöveg, szöveg, szöveg, szöveg, szöveg, szöveg, szöveg, szöveg, szöveg, szöveg,
szöveg, szöveg, szöveg, szöveg, szöveg, szöveg, szöveg, szöveg, szöveg, szöveg, szöveg, szöveg, szöveg, szöveg, szöveg,
szöveg, szöveg, szöveg, szöveg, szöveg, szöveg, szöveg, szöveg, szöveg, szöveg, szöveg, szöveg, szöveg, szöveg, szöveg,
szöveg, szöveg, szöveg, szöveg, szöveg, szöveg, szöveg, szöveg, szöveg, szöveg, szöveg, szöveg, szöveg, szöveg, szöveg,
szöveg, szöveg, szöveg, szöveg, szöveg, szöveg, szöveg, szöveg, szöveg, szöveg, szöveg, szöveg, szöveg, szöveg, szöveg,
szöveg, szöveg, szöveg, szöveg, szöveg, szöveg, szöveg, szöveg, szöveg, szöveg, szöveg, szöveg, szöveg, szöveg, szöveg,
szöveg, szöveg, szöveg, szöveg, szöveg, szöveg, szöveg, szöveg, szöveg, szöveg, szöveg, szöveg, szöveg, szöveg, szöveg,
szöveg, szöveg, szöveg, szöveg, szöveg, szöveg, szöveg, szöveg, szöveg, szöveg, szöveg, szöveg, szöveg, szöveg, szöveg,
szöveg, szöveg, szöveg,  szöveg, szöveg, szöveg, szöveg, szöveg, szöveg, szöveg, szöveg, szöveg, szöveg, szöveg, szöveg,
szöveg, szöveg, szöveg, szöveg, szöveg, szöveg, szöveg, szöveg, szöveg, szöveg, szöveg, szöveg, szöveg, szöveg, szöveg,
szöveg, szöveg, szöveg, szöveg, szöveg, szöveg, szöveg, szöveg, szöveg, szöveg, szöveg, szöveg,
szöveg, szöveg, szöveg, szöveg, szöveg, szöveg, szöveg, szöveg, szöveg, szöveg, szöveg, szöveg, szöveg, szöveg, szöveg,
szöveg, szöveg, szöveg, szöveg, szöveg, szöveg, szöveg, szöveg, szöveg, szöveg, szöveg, szöveg, szöveg, szöveg, szöveg,

\chapter{Egyebek}

\section{Környezetek}
\begin{tet}
\label{tet-alap}
Ez itt egy tetel.
\end{tet}

%A bizonyítás \begin{proof} és \end{proof} közé kerül:
\begin{proof}
Ez pedig a bizonyítása, melyben szerepel egy képlet:
\begin{equation}
\begin{split}
E^{\text{globális}} &= \text{tet}_1\cdot E_1^{\text{elemi}}+\text{tet}_2\cdot
E_2^{\text{elemi}}+\ldots+\text{tet}_n\cdot E_n^{elemi} \\
&=E^{\text{elemi}}\left(\text{tet}_1+\text{tet}_2+\ldots+\text{tet}_n\right)\\
&=E^{\text{elemi}}\cdot\text{össztet}
\end{split}
\end{equation}
A második egyenlõségnél azt használtunk ki, hogy ...

Ezzel a bizonyítást befejeztük.
\end{proof}

\begin{defi}
\label{def-pelda}
Ez egy definíció. Számozása a tetelekkel együtt történik.
\end{defi}

\begin{áll}
A követekezõ négy állítás egymással ekvivalens:
\label{áll-ekvivalencia}
  \begin{itemize}
  \item[(i)] $M$ és $N$ gyengén ekvivalensek.
  \item[(ii)] Minden $n$
  nemnegatív egész számra $|L_{M}\cap \Sigma_{1}^{n}|=|L_{N}\cap \Sigma_{2}^{n}|$ teljesül.
  \item[(iii)] Minden $n$ nemnegatív egész szám esetén
   létezik
  $ \pi_{n}: L_{M}\cap \Sigma_{1}^{n} \rightarrow L_{N}\cap \Sigma_{2}^{n} $ kölcsönösen egyértelmû
  leképezés.
  \item[(iv)] Minden nemnegatív $n$-re $x A^{n} y^{T}=x' A'^{n} y'^{T}$.
  \end{itemize}
\end{áll}

\begin{köv}
  Ez pedig egy következmény.
\end{köv}

\begin{pld}
  Ez lesz a példa, ezt nem szedjük dõlten.
\end{pld}

\begin{megj}
  A fejezetet pedig egy megjegyzés zárja.
\end{megj}


\section{Listák}

Ez egy felsorolás:
\begin{itemize}
    \item elsõ
    \item második
      \subitem elsõ
      \subitem második
    \item harmadik
    \item[$\clubsuit$]  saját jel is alkalmazható
\end{itemize}
Ez pedig egy számozott lista:
\begin{enumerate}
            \item hétfõ
            \item kedd
            \item szerda
\end{enumerate}

%Oldaltörést is alkalmazhatunk
\pagebreak


\section{Egy táblázat és egy ábra}

A táblázat itt következik.
\begin{table}[!h]\label{strategia}
\caption{Példa stratégiatáblára a Black Jack esetében}
\begin{center}
\begin{tabular}{l||r|r|r|r|r|r|r|r|r|r}
&ász&2&3&4&5&6&7&8&9&10\\
\hline\hline
21&n&n&n&n&n&n&n&n&n&n\\
20&n&n&n&n&n&n&n&n&n&n\\
19&n&n&n&n&n&n&n&n&n&n\\
18&n&n&n&n&n&n&n&n&n&n\\
17&n&n&n&n&n&n&n&n&n&n\\
16&h&n&n&n&n&n&h&h&b&b\\
15&h&n&n&n&n&n&h&h&h&b\\
14&h&n&n&n&n&n&h&h&h&b\\
13&h&n&n&n&n&n&h&h&h&h\\
12&h&n&n&n&n&n&h&h&h&h\\
11&h&D&D&D&D&D&D&D&D&h\\
\end{tabular}
\end{center}
\end{table}

Lássunk egy ábrát is!
\begin{figure}[!h]
\unitlength 8mm
\begin{center}
\begin{picture}(8,6)
\thicklines
\multiput(0,1)(0,1){2}{\line(1,0){5}}
\multiput(3,0)(1,0){2}{\line(0,1){6}}
\multiput(1,0)(1,0){2}{\line(0,1){1}}
\multiput(6,0)(1,0){2}{\line(0,1){5}}
\multiput(0,1)(1,0){3}{\line(0,1){1}}
\multiput(2,4)(3,0){3}{\line(0,1){1}}
\multiput(3,0)(0,3){3}{\line(1,0){1}}
\multiput(6,0)(0,1){4}{\line(1,0){1}}
\multiput(7,2)(0,1){2}{\line(1,0){1}}
\multiput(2,4)(0,1){2}{\line(1,0){6}}
\put(5,1){\line(0,1){1}}
\put(8,2){\line(0,1){1}}
\put(1,0){\line(1,0){1}}
\put(1,1){\makebox(1,1){\(\sphericalangle\)}}
\put(7,2){\makebox(1,1){\(\$\)}}
\end{picture}
\end{center}
\caption{\label{labirintus}Labirintus bejárása}
\end{figure}

%laptörés:
\newpage

Külön fájlban elkészített grafika beillesztését a \ref{abra-automata} ábra szemlélteti.
\begin{figure}[h]
\centering
%A psfrag csomag használatával a (encapsulated)postcript abra feliratait LaTeX koddal helyettesíthatjük:
\psfrag{a}[c][c]{$q_0$}
\psfrag{b}[c][c]{$q_1$}
\psfrag{c}[c][c]{$q_2$}
\psfrag{d}[c][c]{$q_3$}
\psfrag{e}[c][c]{$q_4$}
\psfrag{f}[c][c]{$q_5$}
\psfrag{g}[c][c]{$q_6$}
\psfrag{h}[c][c]{$q_7$}
\psfrag{0}[c][c]{$a_{0}$}
\psfrag{9}[c][c]{$a_{9}$}
\psfrag{3}[c][c]{$a_{3}$}
\psfrag{12}[c][c]{$a_{12}$}
\psfrag{15}[c][c]{$a_{15}$}
%Garfika belillesztese, "scale2 a nagyitas/kicinyites merteke, itt 80%.
\includegraphics[scale=0.8]{abra.eps}
\caption{\label{abra-automata} A $4\times m$-es tábla lefedéseinek mátrixreprezentációit felismerõ automata}
\end{figure}


\chapter{Függelék}

\section{A program forráskódja}
A függelékbe kerülhetnek a hosszú táblázatok, vagy mondjuk egy programlista:
% A verbatim kornyezet hasznalatanal ügyeljünk rá, hogy az editor a szóközöjket át ne írja tab karakterekre!
\begin{verbatim}
   while (ujkmodosito[i]<0)
   {
      if (ujkmodosito[i]+kegyenletes[i]<0)
      {
         j=i+1;
         while (j<14)
         if (kegyenletes[i]+ujkmodosito[j]>-1) break;
         else j++;
         temp=ujkmodosito[j];
         for (l=i;l<j;l++) ujkmodosito[l+1]=ujkmodosito[l];
         ujkmodosito[i]=temp;
      }
      i++;
   }
\end{verbatim}


\chapter*{Nyilatkozat}
%Egy üres sort adunk a tartalomjegyzékhez:
\addtocontents{toc}{\ }
\addcontentsline{toc}{section}{Nyilatkozat}
%\hspace{\parindent}

% A nyilatkozat szövege más titkos és nem titkos dolgozatok esetében.
% Csak az egyik tipusú myilatokzatnak kell a dolgozatban szerepelni
% A ponok helyére az adatok értelemszerûen behelyettesídendõk es
% a szakdolgozat /diplomamunka szo megfeleloen kivalasztando.


%A nyilatkozat szövege TITKOSNAK NEM MINÕSÍTETT dolgozatban a következõ:
%A pontokkal jelölt szövegrészek értelemszerûen a szövegszerkesztõben és
%nem kézzel helyettesítendõk:

\noindent
Alulírott \makebox[4cm]{\dotfill} szakos hallgató, kijelentem, hogy a dolgozatomat a Szegedi Tudományegyetem, Informatikai Intézet \makebox[4cm]{\dotfill} Tanszékén készítettem, \makebox[4cm]{\dotfill} diploma megszerzése érdekében.

Kijelentem, hogy a dolgozatot más szakon korábban nem védtem meg, saját munkám eredménye, és csak a hivatkozott forrásokat (szakirodalom, eszközök, stb.) használtam fel.

Tudomásul veszem, hogy szakdolgozatomat / diplomamunkámat a Szegedi Tudományegyetem Informatikai Intézet könyvtárában, a helyben olvasható könyvek között helyezik el.

\vspace*{2cm}

\begin{tabular}{lc}
Szeged, \today\
\hspace{2cm} & \makebox[6cm]{\dotfill} \\
& aláírás \\
\end{tabular}


\vspace*{4cm}

%A nyilatkozat szövege TITKOSNAK MINÕSÍTETT dolgozatban a következõ:

\noindent
Alulírott \makebox[4cm]{\dotfill} szakos hallgató, kijelentem, hogy a dolgozatomat a Szegedi Tudományegyetem, Informatikai Intézet \makebox[4cm]{\dotfill} Tanszékén készítettem, \makebox[4cm]{\dotfill} diploma megszerzése érdekében.

Kijelentem, hogy a dolgozatot más szakon korábban nem védtem meg, saját munkám eredménye, és csak a hivatkozott forrásokat (szakirodalom, eszközök, stb.) használtam fel.

Tudomásul veszem, hogy szakdolgozatomat / diplomamunkámat a TVSZ 4. sz. mellékletében leírtak szerint kezelik.

\vspace*{2cm}

\begin{tabular}{lc}
Szeged, \today\
\hspace{2cm} & \makebox[6cm]{\dotfill} \\
& aláírás \\
\end{tabular}





\chapter*{Köszönetnyilvánítás}
\addcontentsline{toc}{section}{Köszönetnyilvánítás}

Ezúton szeretnék köszönetet mondani \textbf{X. Y-nak} ezért és ezért \ldots


%% Az itrodalomjegyzek keszitheto a BibTeX segedprogrammal:
%\bibliography{diploma}
%\bibliographystyle{plain}

%VAGY "kézzel" a következõ módon:

\begin{thebibliography}{9}
%10-nél kevesebb hivatkozás esetén

%\begin{thebibliography}{99}
% 10-nél több hivatkozás esetén

\addcontentsline{toc}{section}{Irodalomjegyzék}

%Elso szerzok vezetekneve alapjan ábécérendben rendezve.


%folyóirat cikk: szerzok(k), a folyóirat neve kiemelve,
%az evfolyam felkoveren, zarojelben az evszam, vegul az oldalszamok es pont.
\bibitem{Gischer}
J. L. Gischer,
The equational theory of pomsets.
\emph{Theoret. Comput. Sci.}, \textbf{61}(1988), 199--224.

%könyv (szerzo(k), a könyv neve kiemelve, utana a kiado, a kiado szekhelye, az evszam es pont.)
\bibitem{Pin}
J.-E. Pin,
\emph{Varieties of Formal Languages},
Plenum Publishing Corp., New York, 1986.





\end{thebibliography}




\end{document}
