\usepackage{graphicx}\begin{Document}
                         Ezt a szoftvert 2024-ben kezdtem el fejleszteni, de a lényegi részét idén készítettem, folyamatos fejlesztéssel, hibajavítással, kísérletezéssel.


                         Az eredeti beadott verzióval több gond is volt. A programot nem lehetett egy belépési ponttal futtatni. Helyette össze-vissza vibe code-olt, kaotikus, gyakran egymástól teljesen független szkriptek szerepeltek mindenhol, amelyeknek a használatát csak én ismertem, ráadásul egy részük nem is működött, csak úgy "volt". A program egyébként is csak a most "ModelInputPreparer"-nek hívott részt tudta lefuttatni standard Python (pl. PyCharm) környezetben. A Hugging Face modell inferenciája (mostani "HuggingFaceModelInferencer") Jupyter Notebookban futott, Google Colab környezetben, majd a modell outputjának feldolgozása (mostani "ModelOutputProcessor") megint egy harmadik környezetben, Google Apps Scriptben került feldolgozásra.
                         Az újraírásnál első lépésem volt, hogy ezt egyesítem. Most az egész szoftver standard Python futtatókörnyezetben fut, a könnyen megtalálható "main.py" fájlok futtatásával, ipari standard szerint az src mappában lévő globális belépési ponttal, de egyenként is futtatható a 3 modul. PyCharm környezet ajánlott a szoftver futtatásához.
                         A legnagyobb változást a Clean Code című könyv elolvasása hozta, amely hatására átértékeltem a programozói tudásomat, különös tekintettel a szakdolgozatomra. Láttam, hogy objektum orientált módszerrel nagy szoftvereket is átláthatóan és biztonságosan lehet fejleszteni. Megjegyeztem azt is, hogy a jó elnevezések milyen fontosak. Ezek az elvek alapján írtam meg teljesen üres vázlatból az új szakdolgozat szoftveremet.
                         A szoftver "keretrendszer"-ként van hivatkozva, bár én inkább modulnak hívom. 3 fő futtatható állományból áll:

                         A ModelInputPreparer, amely lokális adathalmazból dolgozik. A program az adathalmazból létrehozza a modellnek szánt inputot, a 2800 kérdést. Ez a program alapértelmezetten 'WiC adathalmaz' 'test' splitjének rekordjaiból hozza létre a kérdéseket, de saját, megegyező formátumú kérdésekre is működik. A kérdéseket mind egyenes, mind fordított sorrendben létrehozza, amelyekkel azután a modellek konzisztenciáját vizsgáljuk. Mivel alapértelmezetten 1400 rekordból áll, és rekordonként 2 kérdésünk (egyenes, fordított) van, így jön ki az 2800 kérdés, de ezen igény szerint lehet módosítani.

% TODO
                         A CloudRunnerNotebooks ...


                         A ModelOutputProcessor ...


                         # Használat:
                         \begin{itemize}
                             \item
                         \end{itemize}
                         példa válasz a `Qwen/Qwen2.5-0.5B-Instruct` futtatására:
                         \begin{figure}
                             \centering
                             \includegraphics[keepaspectratio]{pelda}
                             \caption{Példa Válasz}
                             \label{fig:PeldaValasz}
                         \end{figure}


\end{Document}