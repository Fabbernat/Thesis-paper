\cite{pilehvar2019wic}[1] Mohammad Taher Pilehvar és Jose Camacho-Collados. „WiC: the Word-in-Context
Dataset for Evaluating Context-Sensitive Meaning Representations”. Proceedings
of the 2019 Conference of the North American Chapter of the Association for
Computational Linguistics: Human Language Technologies, Volume 1 (Long and
Short Papers). Szerk. Jill Burstein, Christy Doran és Thamar Solorio. Minneapolis,
Minnesota: Association for Computational Linguistics, 2019. jún., 1267–1273. old.
DOI: 10.18653/v1/N19-1128. URL: https://aclanthology.org/
N19-1128/.
\cite{lesk1986automatic}[2] Michael E. Lesk. „Automatic Sense Disambiguation Using Machine Readable Dictionaries: How to Tell a Pine Cone from an Ice Cream Cone”. Proceedings of the
5th Annual International Conference on Systems Documentation (SIGDOC ’86).
New York, NY, USA: ACM, 1986, 24–26. old. URL: https://dl.acm.org/
doi/10.1145/318723.318728.
\cite{sarlin2020superglue}[3] Paul-Edouard Sarlin és tsai. SuperGlue: Learning Feature Matching with Graph Neural Networks. 2020. arXiv: 1911.11763 [cs.CV]. URL: https://
arxiv.org/abs/1911.11763.
\cite{wang2020superglue}[4] Alex Wang és tsai. SuperGLUE: A Stickier Benchmark for General-Purpose Language Understanding Systems. 2020. arXiv: 1905.00537 [cs.CL]. URL: https:
//arxiv.org/abs/1905.00537.
\cite{chiang2024chatbot}[5] Wei-Lin Chiang és tsai. Chatbot Arena: An Open Platform for Evaluating LLMs
by Human Preference. 2024. arXiv: 2403 . 04132 [cs.AI]. URL: https :
//arxiv.org/abs/2403.04132.
\cite{}[6] Springboard. OpenAI GPT-3: Everything You Need to Know. 2023. URL: https:
//www.springboard.com/blog/data-science/machine-learninggpt-3-open-ai/.
\cite{verge2023nytai}[7] The Verge. The New York Times prohibits using its content to train AI models. 2023.
URL: https://www.theverge.com/2023/8/14/23831109/thenew-york-times-ai-web-scraping-rules-terms-of-service.
\cite{tan2024terms}[8] Eli Tan. „When the Terms of Service Change to Make Way for A.I. Training”. The
New York Times (2024. jún.). URL: https://www.nytimes.com/2024/
06/26/technology/terms-service-ai-training.html.
\cite{}[9] Thomas Wolf és tsai. „HuggingFace’s Transformers: State-of-the-art Natural Language Processing”. arXiv preprint arXiv:1910.03771 (2019). URL: https : / /
arxiv.org/abs/1910.03771.
\cite{}[10] Ari Holtzman és tsai. „The Curious Case of Neural Text Degeneration”. Proceedings of the 8th International Conference on Learning Representations (ICLR).
2020. arXiv: 1904 . 09751. URL: https : / / arxiv . org / abs / 1904 .
09751.
\cite{}[11] Gábor Berend. 11\_phi.ipynb. https://colab.research.google.com/
drive/1GQRiTDNWwNPP\_PPARYd1swY1Oiai9Ey\_. Google Colab jegyzetfüzet. 2025. (Elérés dátuma 2025. 05. 23.).
\cite{}[12] Microsoft. Phi-4-mini-instruct. https://huggingface.co/microsoft/
Phi-4-mini-instruct. 2023.
\cite{}[13] Google. Gemma-2-2b-it. https://huggingface.co/google/gemma2-2b-it. 2023.
\cite{}[14] Qwen. Qwen1.5-1.8B-Chat. https://huggingface.co/Qwen/Qwen1.
5-1.8B-Chat. 2023.
\cite{}[15] Steven Bird és Edward Loper. „NLTK: The Natural Language Toolkit”. Proceedings of the ACL Interactive Poster and Demonstration Sessions. Barcelona, Spain:
Association for Computational Linguistics, 2004. júl., 214–217. old. URL: https:
//aclanthology.org/P04-3031/.
\cite{}[16] George A. Miller. „WordNet: A Lexical Database for English”. Human Language
Technology: Proceedings of a Workshop held at Plainsboro, New Jersey, March
8-11, 1994. 1994. URL: https://aclanthology.org/H94-1111/.
\cite{}[17] Tim Peters. PEP 20 – The Zen of Python. https://peps.python.org/
pep-0020/. Python Enhancement Proposal. 2004.
\cite{}[18] Samuel Oloruntoba és Anish Singh Walia. SOLID: The First 5 Principles of Object
Oriented Design. 2024. ápr. URL: https : / / www . digitalocean . com /
community / conceptual - articles / s - o - l - i - d - the - first -
five-principles-of-object-oriented-design.
\cite{}[19] OpenAI. OpenAI API Documentation Overview. 2025. URL: https://platform.
openai.com/docs/overview.
\cite{}[20] Anthropic. Client SDKs - Anthropic API. https://docs.anthropic.com/
en/api/client-sdks. 2025.
\cite{}[21] Alessandro Raganato és tsai. XL-WiC: A Multilingual Benchmark for Evaluating Semantic Contextualization. 2020. arXiv: 2010 . 06478 [cs.CL]. URL:
https://arxiv.org/abs/2010.06478.
\cite{}[22] Anna Breit és tsai. WiC-TSV: An Evaluation Benchmark for Target Sense Verification of Words in Context. 2021. arXiv: 2004.15016 [cs.CL]. URL: https:
//arxiv.org/abs/2004.15016.